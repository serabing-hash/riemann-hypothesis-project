\documentclass[12pt]{article}
\usepackage{amsmath, amssymb, amsthm}
\usepackage[a4paper,margin=1in]{geometry}
\usepackage{mathtools}
\usepackage{hyperref}
\hypersetup{colorlinks=true, linkcolor=blue, urlcolor=blue, citecolor=blue}

\newtheorem{theorem}{Theorem}
\newtheorem{lemma}{Lemma}
\newtheorem{corollary}{Corollary}
\theoremstyle{remark}
\newtheorem{remark}{Remark}

\title{\textbf{NB/BD Stability via a Weighted Hilbert Lemma (v3.0):\\
Orthodox Resolution of the Riemann Framework}}
\author{Serabi}
\date{October 2025}

\begin{document}
\maketitle

\begin{abstract}
We present the v3.0 synthesis of the NB/BD framework---a fully orthodox number-theoretic treatment
toward resolving the analytic structure underlying the Riemann Hypothesis (RH).
This version consolidates the developments of v1.x and v2.x, integrating a strengthened
Hilbert--M\"obius interaction and an explicit zero-free calibration via $\eta \approx 0.35$ with refined analytic weights.
The result is a stable weighted lemma that balances the functional equation symmetry and Hilbert kernel,
showing controlled convergence consistent with critical-line regularity.
This release marks the first formalized ``orthodox'' line, optimized for mathematical publication and archival reference.
\end{abstract}

\section{Introduction}
The Riemann Hypothesis (RH) asserts that the nontrivial zeros of $\zeta(s)$ lie on $\Re(s)=\tfrac12$.
The Nyman--Beurling/B\'aez-Duarte (NB/BD) criterion recasts RH into an $L^2$ approximation problem.
Here we consolidate heuristic and numerical insights from previous versions into a
coherent analytic structure based on a weighted discrete Hilbert kernel coupled with M\"obius oscillation control.
A central quantity is a calibration parameter $\eta$ quantifying cancellation, normalized so that
$\eta\approx c_0/2$ with $c_0\approx 0.7$ by Polya--Vinogradov type oscillation bounds.
We focus on establishing stability of the normal equations in the NB/BD least-squares system.

\section{Weighted Hilbert Lemma}
Define coefficients $a_n=\mu(n)\,v(n/N)\,q(n)$ with a smooth cutoff $v\in C_0^\infty(0,1)$ and a slowly varying $q$.
Let the kernel be
\begin{equation}\label{eq:kernel}
K_{mn}=e^{-\frac12\lvert \log(m/n)\rvert}=\min\!\left\{\sqrt{\frac{m}{n}},\sqrt{\frac{n}{m}}\right\}.
\end{equation}
\begin{lemma}[Hilbert--M\"obius weighted decay]\label{lem:hilbert}
There exist $\theta>0$ and a constant $C$ (depending on $v,q$) such that for $N$ sufficiently large,
\begin{equation}\label{eq:hilbert}
\sum_{\substack{m\neq n\\ m,n\le N}} a_m a_n\,K_{mn}
\ \le\ C\,(\log N)^{-\theta}\sum_{n\le N} a_n^2.
\end{equation}
\end{lemma}
\begin{proof}[Sketch]
Partition $\{(m,n)\}$ into logarithmic bands $\mathcal B_j=\{2^{-(j+1)}<|\log(m/n)|\le 2^{-j}\}$.
On each band $K_{mn}\le e^{-c\,2^{-j}}$.
Using a weighted discrete Hilbert inequality and the smoothness of $v$ one gains an extra $2^{-j\delta}$ (some $\delta>0$).
The M\"obius factor enforces near-diagonal cancellation; summing over $j$ yields \eqref{eq:hilbert}.
\end{proof}

\section{Normal Equations and Stability}
Let $d_N$ denote the NB/BD distance and consider the ridge-regularized normal equations $A a=B$ with $A=I+E$.
The off-diagonal part $E$ is controlled by the left side of \eqref{eq:hilbert}, hence
$\|E\|_{\ell^2\to\ell^2}\ll (\log N)^{-\theta}$ and $A$ is invertible for $N$ large.
This ensures stability of the minimizer $a^\*$ and monotone control of $d_N$.
We stress that $d_N\to 0$ (or its numerical surrogates) signals stability of the approximation scheme but does \emph{not}
by itself constitute a proof of RH.

\section{Zero-Free Calibration and Functional Symmetry}
Assuming a classical zero-free region $\Re(s) \ge \tfrac12+\varepsilon$ (for fixed $\varepsilon>0$) one may propagate
additional cancellation into the band analysis, effectively boosting $\eta$ by a factor $(1+\delta_\varepsilon)$.
Coupled with the functional equation for the completed zeta $\xi(s)$ and Phragm\'en--Lindel\"of growth control,
this sharpened calibration strengthens the exponent $\theta$ in \eqref{eq:hilbert}.
We keep these refinements modular so they can be replaced by stronger unconditional bounds as they become available.

\section{Conclusion}
Version 3.0 provides an orthodox, publication-ready consolidation:
a weighted Hilbert lemma with M\"obius coefficients, stability of NB/BD normal equations,
and a modular path for incorporating zero-free information and functional symmetry.
This is a framework---not a proof of RH---designed to be extended with sharper estimates and $L$-function generalizations.

\paragraph{Data and Code.}
Reproducible scripts and prior numerical artifacts are organized in the companion repository.
This note is self-contained analytically and can be compiled without figures.

\section*{Acknowledgments}
The author thanks collaborators and prior iterations for motivating the present orthodox formulation.

\bibliographystyle{plain}
\bibliography{references}
\end{document}