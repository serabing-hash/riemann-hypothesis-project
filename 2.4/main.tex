\documentclass[11pt]{article}
\usepackage[utf8]{inputenc}
\usepackage[a4paper,margin=1in]{geometry}
\usepackage{amsmath,amssymb,amsthm,mathtools}
\usepackage{hyperref}
\usepackage{booktabs}
\hypersetup{colorlinks=true, linkcolor=blue, urlcolor=blue, citecolor=blue}

\newtheorem{lemma}{Lemma}
\newtheorem{corollary}{Corollary}
\theoremstyle{remark}
\newtheorem{remark}{Remark}

\title{NB/BD Framework Toward RH (v2.4):\\
Orthodox Strengthening via Log--Band/Abel Analysis and Zero--Free Input}
\author{Serabi}
\date{2025}

\begin{document}
\maketitle

\begin{abstract}
We strengthen the orthodox analytic line for the Nyman--Beurling/B\'aez--Duarte (NB/BD) program toward the Riemann Hypothesis.
Compared with v2.3, we provide a more explicit log--band decomposition and discrete Abel summation scheme for M\"obius--weighted coefficients in the Hilbert--type kernel.
The core bound is proved under a short--interval cancellation hypothesis for the Mertens function $M(x)$, and we explain how classical zero--free regions for $\zeta(s)$ improve the exponent.
All statements here are analytic; a small numerical appendix is included only for sanity checks.
\end{abstract}

\section{Setup and Notation}
Write $M(x)=\sum_{n\le x}\mu(n)$.
Fix a smooth cutoff $v\in C_0^\infty(0,1)$ with $\|v^{(k)}\|_\infty\ll_k 1$ and a slowly varying weight $q(n)$ obeying, for all $r\ge 1$,
\begin{equation}\label{eq:qdiff}
|q(n)|\ll (\log N)^C,\qquad \Delta^r q(n)\ll_r (\log N)^C\, n^{-r}.
\end{equation}
Define M\"obius--weighted coefficients
\begin{equation}\label{eq:an}
a_n=\mu(n)\,v\!\Big(\frac{n}{N}\Big)\,q(n),\qquad 1\le n\le N,
\end{equation}
and the Hilbert--type kernel
\begin{equation}\label{eq:K}
K_{mn}=e^{-\frac12|\log(m/n)|}=\min\!\Big\{\sqrt{\frac{m}{n}},\sqrt{\frac{n}{m}}\Big\}.
\end{equation}
The off--diagonal quadratic form is
\begin{equation}\label{eq:S}
S(N):=\sum_{\substack{m\ne n\\ m,n\le N}} a_m a_n\,K_{mn}.
\end{equation}

\section{Short--Interval Hypothesis and Main Lemma}
We formulate a standard short--interval hypothesis for $M(x)$.

\medskip
\noindent\textbf{Hypothesis $H_\eta(\beta)$.} For some $\eta\in(0,1)$ and $\beta\in(0,1]$ there exist $A,C\ge 0$ such that for all $N$ large and all $x\in[N/2,2N]$,
\begin{equation}\label{eq:Heta}
\sup_{|u|\le H}\,|M(x+u)-M(x)|\ \le\ C\, H^{1-\eta}\,(\log N)^A,\qquad H:=N^{\beta}.
\end{equation}

\begin{remark}
Classical zero--free regions for $\zeta(s)$ (see, e.g., Titchmarsh--Heath--Brown) imply versions of \eqref{eq:Heta} with an exponent loss that improves as the zero--free region strengthens; thus $\eta$ may be taken as a positive constant depending on available zero--free data.
Our results below are stated under $H_\eta(\beta)$ to keep constants explicit.
\end{remark}

\begin{lemma}[Weighted Hilbert decay under $H_\eta(\beta)$]\label{lem:decay}
Assume \eqref{eq:qdiff}, \eqref{eq:an} and $H_\eta(\beta)$.
Then there exist constants $C<\infty$ and $\theta=\theta(\eta,\beta,v,q)>0$ such that
\begin{equation}\label{eq:decay}
S(N)\ \le\ C\,(\log N)^{-\theta}\,\sum_{n\le N}|a_n|^2.
\end{equation}
One may take $\theta=\min\{\delta\,\beta,\ \eta\,\beta\}$ for some $\delta=\delta(v,q)>0$ arising from the smooth weight freezing on dyadic log--bands.
\end{lemma}

\begin{proof}[Proof outline]
Partition the index pairs $(m,n)$ into logarithmic bands
\[
\mathcal{B}_j:=\Big\{(m,n):\ 2^{-(j+1)}<|\log(m/n)|\le 2^{-j}\Big\}\qquad (j=0,1,2,\dots,J),
\]
with $J\asymp \log\log N$ chosen so that on $\mathcal{B}_j$ we have $|m-n|\asymp H_j:=2^{-j}N$ and $K_{mn}\le e^{-c\,2^{-j}}$.
Fix a band $j$ and write $m=n+h$ with $|h|\asymp H_j$.
A Taylor/finite--difference expansion using \eqref{eq:qdiff} and smoothness of $v$ gives a frozen weight $W_j(n)$ such that
\[
a_{n+h}\,a_n = \mu(n+h)\mu(n)\,W_j(n)\ +\ O\!\big(2^{-j\delta}\,|a_n|^2\big),
\]
for some $\delta>0$.
Discrete Abel summation in $h$ moves the $\mu$--difference onto $M$:
\[
\sum_{|h|\asymp H_j}\mu(n+h)\, \Delta_h(\cdots)\ =\ \sum_{|h|\asymp H_j} \big(M(n+h)-M(n+h-1)\big)\, \Delta_h(\cdots),
\]
and a telescoping step bounds the partial sums in terms of
\[
\max_{|u|\le H_j}\,|M(n+u)-M(n)|\ \ll\ H_j^{1-\eta}(\log N)^A
\]
by $H_\eta(\beta)$ with $H_j\asymp 2^{-j}N$.
Summing over $n\asymp N$ yields a band contribution
\[
S_j\ \ll\ e^{-c\,2^{-j}}\ \Big(2^{-j\eta}+2^{-j\delta}\Big)\,(\log N)^A \sum_{n\le N}|a_n|^2.
\]
The series $\sum_{j\ge 0} e^{-c\,2^{-j}}2^{-j\min\{\eta,\delta\}}$ converges with a logarithmic saving that can be quantified as $(\log N)^{-\theta}$ once $J\asymp\log\log N$ is tied to $H=N^{\beta}$.
This gives \eqref{eq:decay} with $\theta=\min\{\eta\beta,\delta\beta\}$.
\end{proof}

\begin{corollary}[NB/BD stability under $H_\eta(\beta)$]
Let $A=I+E$ be the normal--equation matrix for the NB/BD minimization.
Then $\|E\|_{\ell^2\to\ell^2}\ll (\log N)^{-\theta}$ with $\theta$ as in Lemma~\ref{lem:decay}.
Hence $A^{-1}$ exists for $N$ large and the optimal distance $d_N\to 0$.
\end{corollary}

\begin{remark}[Unconditional discussion]
Without invoking $H_\eta(\beta)$, explicit--formula methods and classical zero--free regions (Korobov--Vinogradov type) provide subpower savings for $M(x)$ on ranges $H=N^{\beta}$ with $\beta\in(0,1)$.
Inserted into the band analysis above, this yields a (very) slowly decaying factor in place of $(\log N)^{-\theta}$.
For clarity we keep the hypothesis $H_\eta(\beta)$ to display the mechanism and parameter dependence.
\end{remark}

\section{Outlook (v2.5 $\to$ v3.0)}
The path to v3.0 (arXiv submission) is:
\begin{itemize}
\item Insert explicit zero--free constants into \eqref{eq:Heta} (Korobov--Vinogradov), yielding a numerical $\eta(\varepsilon)$ and thus an explicit $\theta$.
\item Track all remainder terms in the freezing/Abel steps to state Lemma~\ref{lem:decay} with named constants depending only on $(v,q)$ and zero--free inputs.
\item Optional: a short appendix linking the NB/BD normal equations to the completed $\xi(s)$ to exploit functional equation symmetry.
\end{itemize}

\appendix
\section*{Appendix A: Single--Band Computation ($j=1$)}
On $\mathcal{B}_1$ we have $|m-n|\asymp H_1\asymp N/2$ and $K_{mn}\le e^{-c/2}$.
Freezing the weight gives $a_{n+h}a_n=\mu(n+h)\mu(n)W_1(n)+O(2^{-\delta}|a_n|^2)$.
Abel summation and $H_\eta(\beta)$ imply
\[
S_1\ \ll\ e^{-c/2}\Big(2^{-\eta}+2^{-\delta}\Big)(\log N)^A\sum_{n\le N}|a_n|^2,
\]
consistent with the general bound.

\section*{Appendix B: Minimal Numerical Check (Illustrative)}
The following small table (not used in proofs) records a sanity check for weighted NB/BD fits at modest sizes.

\begin{center}
\begin{tabular}{@{}rrrr@{}}
\toprule
$N$ & MSE$_+$ & MSE$_-$ & MSE$^\ast=(\text{MSE}_++\text{MSE}_-)/2$\\
\midrule
8000  & 0.118 & 0.208 & 0.163\\
12000 & 0.121 & 0.214 & 0.168\\
16000 & 0.123 & 0.223 & 0.173\\
20000 & 0.122 & 0.218 & 0.170\\
\bottomrule
\end{tabular}
\end{center}

\noindent These values are included only as a record from prior experiments; the present note is analytic.

\begin{thebibliography}{9}
\bibitem{baezduarte2003}
L.~B\'aez--Duarte, \emph{A strengthening of the Nyman--Beurling criterion}, Rend.\ Lincei Mat.\ Appl.\ \textbf{14} (2003), 5--11.

\bibitem{titchmarsh}
E.~C.~Titchmarsh (revised by D.~R.~Heath--Brown), \emph{The Theory of the Riemann Zeta-Function}, 2nd ed., Oxford Univ.\ Press, 1986.

\bibitem{conrey}
J.~B.~Conrey, \emph{The Riemann Hypothesis}, Notices Amer.\ Math.\ Soc.\ \textbf{50} (2003), 341--353.
\end{thebibliography}

\end{document}
