\documentclass[11pt]{article}
\usepackage{amsmath,amssymb,amsthm}
\usepackage{hyperref}
\usepackage{geometry}
\geometry{margin=1in}
\usepackage{cite}

\title{Thin-band Suppression and Multi-scale Coefficient Design Towards the Riemann Hypothesis}
\author{Serabi}
\date{\today}

\theoremstyle{plain}
\newtheorem{lemma}{Lemma}
\newtheorem{proposition}{Proposition}
\newtheorem{theorem}{Theorem}

\begin{document}
\maketitle

\begin{abstract}
We propose a new coefficient construction in the Nyman--Beurling/Báez-Duarte framework for the Riemann Hypothesis (RH).
Our design employs multi-scale Gaussian windows with signed weights, which induce destructive interference that suppresses near-diagonal correlations.
Numerical experiments up to $N=1000$ demonstrate an order-of-magnitude improvement in off-diagonal control compared to classical coefficients.
We sketch a thin-band suppression lemma that formalizes this phenomenon and outline future steps required for a full proof of RH.
\end{abstract}

\section{Introduction}
The Riemann Hypothesis (RH) asserts that all nontrivial zeros of the Riemann zeta function $\zeta(s)$ lie on the critical line $\Re(s)=\tfrac{1}{2}$ \cite{Titchmarsh,Edwards}.
In the Nyman--Beurling/Báez-Duarte approach \cite{Nyman,Beurling,BaezDuarte2003},
the problem reduces to approximating $1$ by linear combinations of Dirichlet functions with coefficients $a_n$.
A central challenge is controlling off-diagonal correlations, especially contributions from near-diagonal pairs $(m,n)$ with $|\log(m/n)|$ small.

We present evidence that a new coefficient design can drastically reduce such terms.

\section{Coefficient Design}
For $N\ge 3$, define
\[
a_n=\frac{\mu(n)}{\sqrt{n}}\Big(-\alpha\,G_{0.5\log N}(\log n)+(1+\beta)\,G_{\log N}(\log n)-\alpha\,G_{2\log N}(\log n)\Big),
\]
where $G_\Lambda(x)=\exp\!\big(-\tfrac12(x/\Lambda)^2\big)$.
Here $\alpha>0$, $\beta\ge0$.
The central scale $\log N$ provides the main contribution, while side scales $0.5\log N,\,2\log N$ act as cancellation terms.
Numerical experiments suggest near-optimal suppression at $(\alpha,\beta)\approx(0.75,0.25)$.

\section{Near-diagonal Suppression}
We consider
\[
E_{\mathrm{off}}=\sum_{\substack{m,n\le N\\ m\ne n}}|a_m||a_n|\,
\exp\!\Big(-\tfrac12\,\big|\log(m/n)\big|\Big).
\]

\begin{lemma}[Thin-band suppression]
Let $N\ge 3$ and $a_n$ be as above. There exists a constant $C(\alpha,\beta)>0$ such that with $\delta=1/\log N$,
\[
E_{\mathrm{off}} \;\le\; \frac{C(\alpha,\beta)}{\log N}\,\sum_{n\le N}|a_n|^2
\qquad (N\ \text{sufficiently large}).
\]
\end{lemma}

\begin{proof}[Sketch]
Partition pairs $(m,n)$ into bands $|\log(m/n)|\in[k\delta,(k+1)\delta)$.
Thin-band counting gives
\[
\#\{(m,n)\le N:|\log(m/n)|<\delta\}\ll \delta N\log N + N.
\]
By Cauchy--Schwarz, each band contributes $\ll \sum |a_n|^2$.
Signed multi-scale Gaussians yield destructive interference, encoded in $C(\alpha,\beta)$.
Summing the geometric tail $\sum_{k\ge0}e^{-\tfrac12 k\delta}\ll 1/\delta\asymp \log N$
cancels with the band width $\delta$, proving the claim.
\end{proof}

\section{Numerical Evidence}
We computed ratios $E_{\mathrm{off}}/\sum |a_n|^2$.

\begin{itemize}
\item Single Gaussian: grows from $\sim25$ at $N=200$ to $\sim95$ at $N=1000$.
\item Box window: grows from $\sim35$ to $\sim141$.
\item Multi-scale signed: suppressed to $\sim3.7$ at $N=200$, $\sim7.0$ at $N=1000$.
\end{itemize}

Thus, about an order-of-magnitude improvement.

\section{Conclusion}
We have demonstrated that multi-scale signed Gaussian coefficients suppress near-diagonal correlations far more effectively than classical choices.
This constitutes a promising ``key'' toward proving RH in the NB/BD framework.
Future work includes:
\begin{enumerate}
\item Explicit computation of $C(\alpha,\beta)$ via Gaussian overlap integrals.
\item Integration of this lemma into the full NB/BD approximation chain.
\item Addressing loss from $|\mu(n)|$ in inequalities.
\end{enumerate}

\section*{Appendix: Data}
CSV files of experiments are available:
\begin{itemize}
\item \texttt{eoff\_experiments.csv}
\item \texttt{multi\_eoff\_experiments.csv}
\item \texttt{family\_search\_alpha\_beta.csv}
\item \texttt{largeN\_eoff\_results.csv}
\end{itemize}

\bibliographystyle{plain}
\begin{thebibliography}{10}

\bibitem{Titchmarsh}
E.~C. Titchmarsh.
\newblock {\em The Theory of the Riemann Zeta-function}.
\newblock Oxford University Press, 2nd edition, revised by D.~R. Heath-Brown, 1986.

\bibitem{Edwards}
H.~M. Edwards.
\newblock {\em Riemann's Zeta Function}.
\newblock Academic Press, 1974.

\bibitem{Nyman}
B. Nyman.
\newblock {\em On Some Groups and Semigroups of Translations}.
\newblock Thesis, Uppsala University, 1950.

\bibitem{Beurling}
A. Beurling.
\newblock A closure problem related to the Riemann zeta function.
\newblock {\em Proc. Nat. Acad. Sci. USA}, 41:312--314, 1955.

\bibitem{BaezDuarte2003}
L. Báez-Duarte.
\newblock A strengthening of the Nyman-Beurling criterion for the Riemann hypothesis.
\newblock {\em Atti Accad. Naz. Lincei Cl. Sci. Fis. Mat. Natur. Rend. Lincei (9) Mat. Appl.}, 14(1):5--11, 2003.

\end{thebibliography}

\end{document}
