\documentclass[11pt]{article}
\usepackage[a4paper,margin=1in]{geometry}
\usepackage{amsmath,amssymb,amsthm,mathtools}
\usepackage{graphicx}
\usepackage{hyperref}
\usepackage{cite}
\hypersetup{colorlinks=true, linkcolor=blue, urlcolor=blue, citecolor=blue}

% --- theorem environments ---
\newtheorem{theorem}{Theorem}
\newtheorem{lemma}{Lemma}
\newtheorem{corollary}{Corollary}
\theoremstyle{remark}
\newtheorem{remark}{Remark}

\title{NB/BD Stability via a Weighted Hilbert Operator: \\ 
Operator--Spectral Roadmap (v3.5)}
\author{Serabi \\ Independent Researcher \\ \texttt{24ping@naver.com}}
\date{2025}

\begin{document}
\maketitle

\begin{abstract}
We refine a weighted Hilbert-operator approach to the Nyman--Beurling/B\'aez-Duarte (NB/BD) criterion.
Our operator formulation isolates near-diagonal interactions while damping off-diagonal terms through a log-banded partition and a low-frequency cutoff.
We provide a self-contained lemma (Hilbert-type decay with M\"obius-weighted coefficients) with explicit assumptions, clarify where numerical heuristics enter, and outline a spectral roadmap (near-normality, compact perturbations, and stability of normal equations).
This note is a \emph{mathematical framework}, not a proof of the Riemann Hypothesis (RH).
\end{abstract}

\section{Setup and Operator Form}
Let $N$ be large. Fix a smooth cutoff $v\in C_0^\infty(0,1)$ with $\|v^{(k)}\|_\infty\ll_k 1$ and a slowly-varying weight $q$ with
\[
|q(n)|\ll(\log N)^C,\qquad \Delta^r q(n)\ll_r (\log N)^C n^{-r}.
\]
Define coefficients $a_n=\mu(n)\,v(n/N)\,q(n)$ and the kernel
\[
K_{mn}=e^{-\frac12|\log(m/n)|}=\min\!\left\{\sqrt{\tfrac{m}{n}},\sqrt{\tfrac{n}{m}}\right\}.
\]
For vectors $x=(x_n)_{n\le N}$ set the discrete operator
\[
(\mathcal{H}x)_m=\sum_{n\le N} K_{mn}\,x_n, \qquad m\le N.
\]
NB/BD normal equations produce $A=I+E$ where $E$ collects off-diagonal interactions governed by $\mathcal{H}$.

\section{Hilbert-Type Decay Lemma}
\begin{lemma}[Weighted Hilbert Decay]
\label{lem:hilbert}
With $a_n=\mu(n)v(n/N)q(n)$ as above, there exist constants $\theta>0$ and $C=C(v,q)$ such that
\[
\sum_{\substack{m\ne n\\ m,n\le N}} a_m a_n\,K_{mn}
\;\le\; C\,(\log N)^{-\theta}\sum_{n\le N} a_n^2.
\]
\end{lemma}

\begin{proof}[Sketch]
Partition pairs $(m,n)$ into dyadic log-bands
\(
\mathcal{B}_j:=\{(m,n):2^{-(j+1)}<|\log(m/n)|\le 2^{-j}\}.
\)
On $\mathcal{B}_j$ we have $K_{mn}\le e^{-c\,2^{-j}}$.
A discrete Hilbert-type bound with smooth cutoffs controls the raw band sum by $(\log N)\|a\|_2^2$.
With $a_n=\mu(n)\cdot$(low-frequency), summation by parts across each band gains an extra $2^{-j\delta}$ via smoothness of $v$ and the oscillation of $\mu(n)$, yielding
\[
\sum_{(m,n)\in\mathcal{B}_j} a_m a_n K_{mn}
\;\ll\; e^{-c\,2^{-j}} (2^{-j}\log N)^{1-\varepsilon}\sum a_n^2.
\]
Summing $j$ gives the stated $(\log N)^{-\theta}$ suppression for some $\theta>0$.
\end{proof}

\begin{remark}[Scope]
Lemma~\ref{lem:hilbert} is qualitative: it ensures \emph{some} $\theta>0$ under the stated smooth and low-frequency assumptions.
Sharper, explicit $\theta$ requires zero-free input or stronger mean value bounds for M\"obius correlations; this is outside the present note.
\end{remark}

\section{Spectral Roadmap}
Write $A=I+E$ for the NB/BD normal matrix.
Lemma~\ref{lem:hilbert} gives $\|E\|_{\ell^2\to \ell^2}\le C(\log N)^{-\theta}<1$ for $N$ large, so $A$ is invertible via a Neumann series.
This supports stability of the least-squares coefficients and the NB/BD distance $d_N$ in the weighted, low-frequency regime.
A rigorous spectral path forward is:
\begin{itemize}
\item[(i)] \textbf{Near-normality:} show $[\mathcal{H},\mathcal{H}^\ast]$ is compact/small on the weighted subspace;
\item[(ii)] \textbf{Band-limited compactness:} the off-diagonal tail is compact under log-banding and smoothing;
\item[(iii)] \textbf{Fredholm alternative:} stability of $A=I+E$ under compact perturbations yields control of minimizers.
\end{itemize}

\section{Limitations and Outlook}
Our analysis does not prove RH. It isolates why the NB/BD linear system is \emph{stable} under weighted, low-frequency designs.
Future steps (all purely NT):
\begin{itemize}
\item explicit band-by-band constants and an effective $\theta(\delta)$;
\item zero-free region input to trade oscillation for quantitative decay;
\item functional equation/explicit formula integration to close remaining gaps.
\end{itemize}

\begin{thebibliography}{9}
\bibitem{BaezDuarte2003}
L.~B\'aez-Duarte,
\emph{A strengthening of the Nyman--Beurling criterion for the Riemann Hypothesis},
Rend. Lincei Mat. Appl. \textbf{14} (2003), 5--11.

\bibitem{Titchmarsh}
E.~C. Titchmarsh (rev. D.~R. Heath-Brown),
\emph{The Theory of the Riemann Zeta-Function}, 2nd ed., OUP, 1986.

\bibitem{Conrey}
J.~B. Conrey,
\emph{The Riemann Hypothesis},
Notices AMS \textbf{50} (2003), 341--353.
\end{thebibliography}

\end{document}