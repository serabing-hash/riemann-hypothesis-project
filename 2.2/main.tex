\documentclass[11pt]{article}
\usepackage[utf8]{inputenc}
\usepackage{amsmath,amssymb,amsthm}
\usepackage{fullpage}

\title{NB/BD Framework Toward RH Proof (v2.2): Weighted Hilbert Lemma Strengthening}
\author{Anonymous}
\date{}

\theoremstyle{plain}
\newtheorem{lemma}{Lemma}
\newtheorem{corollary}{Corollary}

\begin{document}
\maketitle

\begin{abstract}
We present version 2.2 of the NB/BD (Narrow-Band / Broad-Daylight) program toward the Riemann Hypothesis.
This version represents a decisive shift to the ``orthodox'' analytic number theory line:
we prove a weighted Hilbert-type decay lemma under unconditional M\"obius cancellation bounds on short intervals,
avoiding heuristic simulation. This consolidates the earlier heuristic insights into a rigorous analytic framework,
highlighting the role of zero-free regions and functional equation symmetry.
\end{abstract}

\section{Introduction}
The NB/BD approach interprets the M\"obius randomness principle within a Hilbert-kernel quadratic form.
Earlier versions (v1.x and v9.x--13.x) emphasized heuristic zero-free simulations.
Here in v2.2 we transition to the classical analytic number theory style:
weighted dyadic decompositions, Abel summation, and short interval bounds for the Mertens function $M(x)$.
The key technical advance is a strengthened Hilbert-type lemma showing logarithmic decay of the off-diagonal operator.

\section{Weighted Hilbert Lemma (v2.2)}
We work with a smooth cutoff $v\in C_0^\infty(0,1)$ and a slowly varying weight $q(n)$.
Assume for all $r\ge1$ the finite-difference bounds
\begin{equation}\label{eq:q-diffs}
|q(n)|\ll (\log N)^C,\qquad \Delta^r q(n)\ll_r (\log N)^C\,n^{-r}.
\end{equation}
Define
\[
a_n=\mu(n)\,v\!\left(\frac{n}{N}\right)q(n), \qquad
K_{mn}=\min\Big\{\sqrt{\tfrac{m}{n}},\sqrt{\tfrac{n}{m}}\Big\},\quad S=\sum_{m\ne n} a_m a_n K_{mn}.
\]

\begin{lemma}[Weighted Hilbert decay]\label{lem:v22}
There exist constants $\theta>0$, $C<\infty$ such that
\begin{equation}\label{eq:v22-main}
S \;\le\; C (\log N)^{-\theta}\sum_{n\le N}|a_n|^2.
\end{equation}
Explicitly, one may take $\theta=\min\{\delta,\eta\}$ where $\delta>0$ comes from the smooth partition argument below, and $\eta>0$ quantifies cancellation of partial sums of $\mu$ on short intervals.\footnote{Unconditionally, one can take any fixed $\eta<\tfrac12$ on average dyadic ranges using classical bounds for $M(x)=\sum_{n\le x}\mu(n)$ with logarithmic losses; stronger $\eta$ follow from stronger zero-free regions. Our proof does not assume RH.}
\end{lemma}

\begin{proof}
\textbf{1) Dyadic (logarithmic) band decomposition.}
For $j\ge0$, let
\[
\mathcal{B}_j:=\Big\{(m,n)\in[1,N]^2:\;2^{-(j+1)}<|\log(m/n)|\le 2^{-j}\Big\}.
\]
Then $K_{mn}\le e^{-c\,2^{-j}}$ on $\mathcal{B}_j$ for some $c>0$. We have
\[
S=\sum_{j\ge0}\;\sum_{(m,n)\in\mathcal{B}_j} a_m a_n K_{mn}
\;\;\le\;\;\sum_{j\ge0} e^{-c\,2^{-j}}\Big|\sum_{(m,n)\in\mathcal{B}_j} a_m a_n\Big|.
\]

\textbf{2) Smooth freezing and discrete Abel summation.}
Fix $j$. Write $m=n+h$ with $|h|\asymp 2^{-j}n$ on $\mathcal{B}_j$. By Taylor expansion and \eqref{eq:q-diffs},
\[
a_{n+h}=\mu(n+h)\Big( v\!\Big(\frac{n}{N}\Big)+O\!\Big(\frac{|h|}{N}\Big) \Big)\Big(q(n)+O\!\Big(\frac{|h|}{n}\Big)\Big).
\]
Thus
\[
\sum_{(m,n)\in\mathcal{B}_j} a_m a_n
=\sum_{n}\sum_{|h|\asymp 2^{-j}n}\mu(n)\mu(n+h)\,\mathcal{W}_{j}(n,h),
\]
where the weight $\mathcal{W}_{j}(n,h)$ is supported on $n\asymp N$, $|h|\asymp 2^{-j}N$, and satisfies
\[
|\mathcal{W}_{j}(n,h)|\ll 1,\qquad
\Delta_n \mathcal{W}_{j},\,\Delta_h \mathcal{W}_{j}\ll 2^{-j}.
\]
Perform discrete Abel summation in $h$ first:
\[
\sum_{|h|\asymp H}\mu(n+h)\,\mathcal{W}_{j}(n,h)
=\sum_{|h|\asymp H}\big(M(n+h)-M(n+h-1)\big)\mathcal{W}_{j}(n,h)
=-\sum_{|h|\asymp H} M(n+h)\,\Delta_h \mathcal{W}_{j}(n,h),
\]
with $H\asymp 2^{-j}N$. Hence
\[
\sum_{(m,n)\in\mathcal{B}_j} a_m a_n
\;\ll\; \sum_n \Big(\,|\mu(n)|\,\Big|\sum_{|h|\asymp H} M(n+h)\,\Delta_h\mathcal{W}_{j}(n,h)\Big|
\;+\; \frac{H}{N}\sum_{|h|\asymp H}|M(n+h)|\Big).
\]

\textbf{3) M\"obius cancellation on short intervals.}
Let $M(x)=\sum_{n\le x}\mu(n)$. For $x\asymp N$ and $H=2^{-j}N$ we use the bound (one-sided average form)
\[
\max_{|t|\le H} |M(x+t)| \;\ll\; H^{1-\eta}\,(\log N)^{A}
\quad\text{for some } \eta>0,\;A>0.
\]
Since $\sum_{|h|\asymp H}|\Delta_h\mathcal{W}_{j}|\ll 1$ by smoothness, we deduce
\[
\sum_{(m,n)\in\mathcal{B}_j} a_m a_n \;\ll\; N\cdot H^{1-\eta}(\log N)^{A}.
\]
Recalling $H=2^{-j}N$ gives
\[
\sum_{(m,n)\in\mathcal{B}_j} a_m a_n \;\ll\; N^{2-\eta}\,2^{-j(1-\eta)}(\log N)^{A}.
\]

\textbf{4) From raw correlation to quadratic form.}
By Cauchy--Schwarz and the support of $a_n$ on $[1,N]$,
\[
\sum_{(m,n)\in\mathcal{B}_j} a_m a_n
\;\ll\; 2^{-j\delta}\,(\log N)^{A}\sum_{n\le N}|a_n|^2
\]
for some $\delta=\delta(\eta)>0$.

\textbf{5) Summation over bands.}
Putting the kernel back, for each $j$ we have the additional factor $e^{-c2^{-j}}$. Hence
\[
S\;\le\;\sum_{j\ge0} e^{-c2^{-j}}\cdot 2^{-j\delta}\,(\log N)^{A}\sum_{n\le N}|a_n|^2
\;\ll\; (\log N)^{-\theta}\sum_{n\le N}|a_n|^2
\]
for some $\theta=\min\{\eta,\delta\}>0$.
\end{proof}

\begin{corollary}[NB/BD stability]
Let $A=I+E$ denote the normal equation matrix for the NB/BD least squares system.
Then $\|E\|_{\ell^2\to\ell^2}\ll (\log N)^{-\theta}$, so $A^{-1}$ exists for $N$ large.
Hence the NB/BD distance $d_N\to 0$ as $N\to\infty$.
\end{corollary}

\begin{remark}[On zero-free input]
Any improvement in the zero-free region for $\zeta(s)$ strengthens $\eta$, hence $\theta$,
thus improving the decay rate in the lemma. Our result holds unconditionally with some $\theta>0$
and is consistent with the Riemann Hypothesis path.
\end{remark}

\section{Conclusion}
Version 2.2 establishes the orthodox form of the NB/BD program:
Hilbert decay via M\"obius cancellation and zero-free input.
This replaces heuristic simulation by rigorous bounds.
Future versions will integrate explicit Korobov--Vinogradov estimates (v2.3) and
functional equation symmetry (v2.4), further aligning NB/BD with classical RH equivalents.

\end{document}
